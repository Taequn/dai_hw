\documentclass{article}\usepackage[]{graphicx}\usepackage[]{color}
% maxwidth is the original width if it is less than linewidth
% otherwise use linewidth (to make sure the graphics do not exceed the margin)
\makeatletter
\def\maxwidth{ %
  \ifdim\Gin@nat@width>\linewidth
    \linewidth
  \else
    \Gin@nat@width
  \fi
}
\makeatother

\definecolor{fgcolor}{rgb}{0.345, 0.345, 0.345}
\newcommand{\hlnum}[1]{\textcolor[rgb]{0.686,0.059,0.569}{#1}}%
\newcommand{\hlstr}[1]{\textcolor[rgb]{0.192,0.494,0.8}{#1}}%
\newcommand{\hlcom}[1]{\textcolor[rgb]{0.678,0.584,0.686}{\textit{#1}}}%
\newcommand{\hlopt}[1]{\textcolor[rgb]{0,0,0}{#1}}%
\newcommand{\hlstd}[1]{\textcolor[rgb]{0.345,0.345,0.345}{#1}}%
\newcommand{\hlkwa}[1]{\textcolor[rgb]{0.161,0.373,0.58}{\textbf{#1}}}%
\newcommand{\hlkwb}[1]{\textcolor[rgb]{0.69,0.353,0.396}{#1}}%
\newcommand{\hlkwc}[1]{\textcolor[rgb]{0.333,0.667,0.333}{#1}}%
\newcommand{\hlkwd}[1]{\textcolor[rgb]{0.737,0.353,0.396}{\textbf{#1}}}%
\let\hlipl\hlkwb

\usepackage{framed}
\makeatletter
\newenvironment{kframe}{%
 \def\at@end@of@kframe{}%
 \ifinner\ifhmode%
  \def\at@end@of@kframe{\end{minipage}}%
  \begin{minipage}{\columnwidth}%
 \fi\fi%
 \def\FrameCommand##1{\hskip\@totalleftmargin \hskip-\fboxsep
 \colorbox{shadecolor}{##1}\hskip-\fboxsep
     % There is no \\@totalrightmargin, so:
     \hskip-\linewidth \hskip-\@totalleftmargin \hskip\columnwidth}%
 \MakeFramed {\advance\hsize-\width
   \@totalleftmargin\z@ \linewidth\hsize
   \@setminipage}}%
 {\par\unskip\endMakeFramed%
 \at@end@of@kframe}
\makeatother

\definecolor{shadecolor}{rgb}{.97, .97, .97}
\definecolor{messagecolor}{rgb}{0, 0, 0}
\definecolor{warningcolor}{rgb}{1, 0, 1}
\definecolor{errorcolor}{rgb}{1, 0, 0}
\newenvironment{knitrout}{}{} % an empty environment to be redefined in TeX

\usepackage{alltt}
\usepackage{amsmath} %This allows me to use the align functionality.
                     %If you find yourself trying to replicate
                     %something you found online, ensure you're
                     %loading the necessary packages!
\usepackage{amsfonts}%Math font
\usepackage{graphicx}%For including graphics
\usepackage{hyperref}%For Hyperlinks
\hypersetup{colorlinks = true,citecolor=black}
\usepackage{natbib}        %For the bibliography
\bibliographystyle{apalike}%For the bibliography
\usepackage[margin=1.0in]{geometry}
\usepackage{float}
\IfFileExists{upquote.sty}{\usepackage{upquote}}{}
\begin{document}
\noindent \textbf{MA 354: Data Analysis I -- Fall 2021}\\%\\ gives you a new line
\noindent \textbf{Homework 3:}\vspace{1em}\\
\emph{Complete the following opportunities to use what we've talked about in class. These questions will be graded for correctness, communication and succinctness. Ensure you show your work and explain your logic in a legible and refined submission.}
%Comments -- anything after % is not put into the PDF
\begin{enumerate}
%%%%%%%%%%%%%%%%%%%%%%%%%%%%%%%%%%%%%%%%%%%%%%%%%%%%%%%%%%%%%%%%%%%%%%%%%%%%%%%
%%%%%%%%%%%%%%%%%%%%%%%%%%%%%%%%%%%%%%%%%%%%%%%%%%%%%%%%%%%%%%%%%%%%%%%%%%%%%%%
%%%%%%%%%  Question 0
%%%%%%%%%%%%%%%%%%%%%%%%%%%%%%%%%%%%%%%%%%%%%%%%%%%%%%%%%%%%%%%%%%%%%%%%%%%%%%%
%%%%%%%%%%%%%%%%%%%%%%%%%%%%%%%%%%%%%%%%%%%%%%%%%%%%%%%%%%%%%%%%%%%%%%%%%%%%%%%
\item[0.] \textbf{Complete weekly diagnostics.}
  
%%%%%%%%%%%%%%%%%%%%%%%%%%%%%%%%%%%%%%%%%%%%%%%%%%%%%%%%%%%%%%%%%%%%%%%%%%%%%%%
%%%%%%%%%%%%%%%%%%%%%%%%%%%%%%%%%%%%%%%%%%%%%%%%%%%%%%%%%%%%%%%%%%%%%%%%%%%%%%%
%%%%%%%%%  Question 1
%%%%%%%%%%%%%%%%%%%%%%%%%%%%%%%%%%%%%%%%%%%%%%%%%%%%%%%%%%%%%%%%%%%%%%%%%%%%%%%
%%%%%%%%%%%%%%%%%%%%%%%%%%%%%%%%%%%%%%%%%%%%%%%%%%%%%%%%%%%%%%%%%%%%%%%%%%%%%%%
\item Consider  data originally from a study of the nesting horseshoe crabs 
\citep{Brockmann96}. Each female crab in the study had a male crab attached to 
her in her nest. The study investigated factors that affect whether the female 
crab had any other males, called satellites, residing nearby her. Explanatory 
variables thought possibly to affect this included the female crab's color, 
spine condition, weight, and carapace width. The response outcome for each 
female crab is her number of satellites.
	 The sample is
	\begin{center}
		\begin{tabular}{|ccccccccccccccccc|}\hline
		 Number of Satellites& 0 & 1&  2&  3&  4&  5&  6&  7&  8&  9& 10& 11& 12& 13 &14& 15\\ \hline
		 Number of Observations&62& 16&  9& 19& 19& 15& 13&  4&  6&  3&  3&  1&  1&  1&  1&1\\\hline
 		\end{tabular}
	\end{center}
 	It is believed that the distribution of the number of satellites for a female 
 	crab is distributed Poisson$(\lambda)$ where the parameter $\lambda$ is of interest.
	\begin{enumerate}
	\item Calculate the method of moments estimator for $\lambda$.\\
	\textbf{Solution:} We'll be using a nleqslv package \cite{nlesqlv} to solve a non-linear equation to find $\lambda$! Hopefully, you're excited!
\begin{knitrout}
\definecolor{shadecolor}{rgb}{0.969, 0.969, 0.969}\color{fgcolor}\begin{kframe}
\begin{alltt}
\hlkwd{library}\hlstd{(nleqslv)}
\hlkwd{library}\hlstd{(tidyverse)}
\hlcom{#put data into the dataframe}
\hlstd{crab.dat} \hlkwb{<-} \hlkwd{data.frame}\hlstd{(}\hlkwc{x}\hlstd{=}\hlkwd{c}\hlstd{(}\hlkwd{rep}\hlstd{(}\hlnum{0}\hlstd{,}\hlnum{62}\hlstd{),} \hlkwd{rep}\hlstd{(}\hlnum{1}\hlstd{,} \hlnum{16}\hlstd{),} \hlkwd{rep}\hlstd{(}\hlnum{2}\hlstd{,} \hlnum{9}\hlstd{),} \hlkwd{rep}\hlstd{(}\hlnum{3}\hlstd{,} \hlnum{19}\hlstd{),}
                           \hlkwd{rep}\hlstd{(}\hlnum{4}\hlstd{,} \hlnum{19}\hlstd{),} \hlkwd{rep}\hlstd{(}\hlnum{5}\hlstd{,} \hlnum{15}\hlstd{),} \hlkwd{rep}\hlstd{(}\hlnum{6}\hlstd{,} \hlnum{13}\hlstd{),} \hlkwd{rep}\hlstd{(}\hlnum{7}\hlstd{,} \hlnum{4}\hlstd{),}
                           \hlkwd{rep}\hlstd{(}\hlnum{8}\hlstd{,} \hlnum{6}\hlstd{),} \hlkwd{rep}\hlstd{(}\hlnum{9}\hlstd{,} \hlnum{3}\hlstd{),} \hlkwd{rep}\hlstd{(}\hlnum{10}\hlstd{,} \hlnum{3}\hlstd{),} \hlnum{11}\hlstd{,} \hlnum{12}\hlstd{,}
                           \hlnum{13}\hlstd{,} \hlnum{14}\hlstd{,} \hlnum{15}\hlstd{))}



\hlcom{#MOM function}
\hlstd{momfunc} \hlkwb{<-} \hlkwa{function}\hlstd{(}\hlkwc{par}\hlstd{,} \hlkwc{data}\hlstd{)\{}
  \hlstd{population.mean} \hlkwb{<-} \hlstd{par}
  \hlstd{sample.mean} \hlkwb{<-} \hlkwd{mean}\hlstd{(data)}
  \hlkwd{return}\hlstd{(sample.mean}\hlopt{-}\hlstd{population.mean)}
\hlstd{\}}
\hlcom{#Solve the equation}
\hlstd{answer} \hlkwb{<-} \hlkwd{nleqslv}\hlstd{(}\hlkwc{x}\hlstd{=}\hlnum{1}\hlstd{,} \hlcom{#worst guess :3}
        \hlkwc{fn} \hlstd{= momfunc,}
        \hlkwc{data}\hlstd{=crab.dat}\hlopt{$}\hlstd{x)}

\hlkwd{paste}\hlstd{(}\hlstr{"Lambda is ~"}\hlstd{, answer}\hlopt{$}\hlstd{x,} \hlkwc{sep}\hlstd{=}\hlstr{""}\hlstd{)}
\end{alltt}
\begin{verbatim}
## [1] "Lambda is ~2.97701149425287"
\end{verbatim}
\end{kframe}
\end{knitrout}
	\item Find the maximum likelihood estimator for $\lambda$.\\
	\textbf{Solution:} For this, we'll use a \texttt{dpois()} function from the base R package and \texttt{optim()} function to find MLE for our Poisson distribution.
\begin{knitrout}
\definecolor{shadecolor}{rgb}{0.969, 0.969, 0.969}\color{fgcolor}\begin{kframe}
\begin{alltt}
\hlstd{LLfunc} \hlkwb{<-} \hlkwa{function}\hlstd{(}\hlkwc{par}\hlstd{,} \hlkwc{data}\hlstd{)\{}
  \hlstd{answer} \hlkwb{<-} \hlkwd{sum}\hlstd{(}\hlkwd{dpois}\hlstd{(}\hlkwc{x}\hlstd{=data,} \hlkwc{lambda} \hlstd{= par,} \hlkwc{log}\hlstd{=T))}
  \hlopt{-}\hlstd{answer} \hlcom{#most likely}
\hlstd{\}}

\hlstd{answer} \hlkwb{<-} \hlkwd{optim}\hlstd{(}\hlkwc{par} \hlstd{=} \hlnum{1}\hlstd{,} \hlcom{#worst guess c:}
      \hlkwc{fn} \hlstd{= LLfunc,}
      \hlkwc{data} \hlstd{= crab.dat}\hlopt{$}\hlstd{x,}
      \hlkwc{method} \hlstd{=} \hlstr{"Brent"}\hlstd{,}
      \hlkwc{lower} \hlstd{=} \hlnum{0}\hlstd{,}
      \hlkwc{upper} \hlstd{=} \hlnum{100}\hlstd{)}

\hlkwd{paste}\hlstd{(}\hlstr{"Parameter:"}\hlstd{,answer}\hlopt{$}\hlstd{par)}
\end{alltt}
\begin{verbatim}
## [1] "Parameter: 2.9770114489889"
\end{verbatim}
\end{kframe}
\end{knitrout}
	So $\lambda$ is 2.97! It will help us when plotting the data with fitted Poisson distribution over it!
	\item Plot the data with the Poisson distribution fit with the MLE estimates. 
	How well does the distribution fit the data? 
\begin{knitrout}
\definecolor{shadecolor}{rgb}{0.969, 0.969, 0.969}\color{fgcolor}\begin{kframe}
\begin{alltt}
        \hlstd{crab.dat.plot} \hlkwb{<-} \hlstd{crab.dat} \hlopt
          \hlkwd{mutate}\hlstd{(}\hlkwc{f} \hlstd{=} \hlkwd{dpois}\hlstd{(}\hlkwc{x}\hlstd{=x,} \hlkwc{lambda}\hlstd{=}\hlnum{2.977011}\hlstd{))}


        \hlstd{sats} \hlkwb{<-} \hlnum{0}\hlopt{:}\hlnum{15}
        \hlkwd{ggplot}\hlstd{(crab.dat.plot,} \hlkwd{aes}\hlstd{(}\hlkwc{x}\hlstd{=x))}\hlopt{+}
          \hlkwd{geom_histogram}\hlstd{(}\hlkwd{aes}\hlstd{(}\hlkwc{y}\hlstd{=..density..),} \hlkwc{fill}\hlstd{=}\hlstr{"deepskyblue2"}\hlstd{,} \hlkwc{color}\hlstd{=}\hlstr{"black"}\hlstd{,}
                         \hlkwc{binwidth}\hlstd{=}\hlnum{1}\hlstd{)}\hlopt{+}
          \hlkwd{geom_linerange}\hlstd{(}\hlkwd{aes}\hlstd{(}\hlkwc{ymin}\hlstd{=}\hlnum{0}\hlstd{,} \hlkwc{ymax}\hlstd{=f),} \hlkwc{color}\hlstd{=}\hlstr{"dark red"}\hlstd{)}\hlopt{+}
          \hlkwd{scale_x_continuous}\hlstd{(}\hlstr{"Number of satellites"}\hlstd{,}
                             \hlkwc{labels} \hlstd{=} \hlkwd{as.character}\hlstd{(sats),} \hlkwc{breaks} \hlstd{= sats)}\hlopt{+}
          \hlkwd{labs}\hlstd{(}\hlkwc{y}\hlstd{=}\hlstr{"Density"}\hlstd{,}
               \hlkwc{title}\hlstd{=}\hlstr{"Number of observations per satellite"}\hlstd{,}
               \hlkwc{subtitle}\hlstd{=}\hlstr{"Poisson distribution. Lambda=2.977"}\hlstd{)}\hlopt{+}
  \hlkwd{theme_bw}\hlstd{()}\hlopt{+}
  \hlkwd{geom_hline}\hlstd{(}\hlkwc{yintercept}\hlstd{=}\hlnum{0}\hlstd{)}
\end{alltt}
\end{kframe}
\end{knitrout}
	
	\begin{figure}[H]
\begin{center}
\begin{knitrout}
\definecolor{shadecolor}{rgb}{0.969, 0.969, 0.969}\color{fgcolor}
\includegraphics[width=\maxwidth]{figure/unnamed-chunk-3-1} 
\end{knitrout}
	\caption{Poisson model fitted to our data}
\label{plot3} %we can now reference plot1
\end{center}
\end{figure}
As you might notice, our model doesn't really fit the data that neatly — it doesn't take into account the number of zeros that we have in our dataset. This leads the distribution to either underestimate or overestimate values!

	\item Let's try another distribution - the zero-inflated Poisson distribution. 
	Now, it is believed that the distribution of the number of satellites for a 
	female crab is distributed Poisson$_0(\lambda,\sigma)$ where the parameters 
	$\lambda$ and $\sigma$ are of interest. Find the method of moments estimators 
	for both $\sigma$ and $\lambda$.
	 \[f_X(x|\lambda,\sigma)=(1-\sigma)\frac{\lambda^x e^{-\lambda}}{x!}I(x\geq1) 
	 + (\sigma+(1-\sigma)e^{-\lambda}) I(x=0)\]
	 \textbf{Hint:} Noticing that this is a function of the Poisson PMF and
	 rewriting it will help remarkably.\\
	 \textbf{Solution:} Let's use our favorite nleqslv package \cite{nlesqlv} to calculate the parameters that we are looking for for our zero-inflated Poisson model. We need to find $\lambda$ and $\sigma$!\\
	 When calculating E(X) for ZIP, I noticed that the only difference between the regular Poisson distribution and this one is an additional $(1-\sigma)$, so I assumed that E(X)=$(1-\sigma)\lambda$. While I managed to calculate it, I struggle with Var(X), so I found \cite{b_minerb} discussion on StackOverflow really helpful for my calculations!
\begin{knitrout}
\definecolor{shadecolor}{rgb}{0.969, 0.969, 0.969}\color{fgcolor}\begin{kframe}
\begin{alltt}
\hlstd{zip.pois} \hlkwb{<-} \hlkwa{function}\hlstd{(}\hlkwc{par}\hlstd{,} \hlkwc{data}\hlstd{)\{}
  \hlstd{lambda} \hlkwb{<-} \hlstd{par[}\hlnum{1}\hlstd{]}
  \hlstd{pi} \hlkwb{<-} \hlstd{par[}\hlnum{2}\hlstd{]}

  \hlcom{#Applying my knowledge}
  \hlstd{EX1} \hlkwb{<-} \hlstd{(}\hlnum{1}\hlopt{-}\hlstd{pi)}\hlopt{*}\hlstd{lambda}
  \hlstd{EX2} \hlkwb{<-} \hlstd{(}\hlnum{1}\hlopt{-}\hlstd{pi)}\hlopt{*}\hlstd{(lambda}\hlopt{^}\hlnum{2}\hlopt{+}\hlstd{lambda)}

  \hlstd{xbar1} \hlkwb{<-} \hlkwd{mean}\hlstd{(data)}
  \hlstd{xbar2} \hlkwb{<-} \hlkwd{mean}\hlstd{(data}\hlopt{^}\hlnum{2}\hlstd{)}

  \hlkwd{c}\hlstd{(EX1}\hlopt{-}\hlstd{xbar1, EX2}\hlopt{-}\hlstd{xbar2)}
\hlstd{\}}

\hlcom{#solving}
\hlstd{answerZIP}\hlkwb{<-}\hlkwd{nleqslv}\hlstd{(}\hlkwc{x} \hlstd{=} \hlkwd{c}\hlstd{(}\hlnum{1}\hlstd{,}\hlnum{0}\hlstd{),} \hlcom{#best guess at parameter}
        \hlkwc{fn} \hlstd{= zip.pois,}
        \hlkwc{data} \hlstd{= crab.dat}\hlopt{$}\hlstd{x)}

\hlkwd{paste}\hlstd{(}\hlstr{"Lambda:"}\hlstd{,answerZIP}\hlopt{$}\hlstd{x[}\hlnum{1}\hlstd{],}\hlstr{"Sigma:"}\hlstd{,answerZIP}\hlopt{$}\hlstd{x[}\hlnum{2}\hlstd{])}
\end{alltt}
\begin{verbatim}
## [1] "Lambda: 5.46332046334072 Sigma: 0.455091182323775"
\end{verbatim}
\end{kframe}
\end{knitrout}

It's impossible to assess how accurate we are in our calculations without graphing the results. I am going to use \texttt{dzipois()} from VGAM package \cite{VGAM}. 

\begin{figure}[H]
\begin{center}
\begin{knitrout}
\definecolor{shadecolor}{rgb}{0.969, 0.969, 0.969}\color{fgcolor}\begin{kframe}
\begin{alltt}
\hlkwd{library}\hlstd{(}\hlstr{"VGAM"}\hlstd{)}
\hlstd{crab.dat.zim} \hlkwb{<-} \hlstd{crab.dat} \hlopt
          \hlkwd{mutate}\hlstd{(}\hlkwc{f} \hlstd{=} \hlkwd{dzipois}\hlstd{(}\hlkwc{x}\hlstd{=x,} \hlkwc{lambda}\hlstd{=}\hlnum{5.4633205}\hlstd{,} \hlkwc{pstr0}\hlstd{=}\hlnum{0.4550912}\hlstd{))}
\hlcom{#make a true density function}

\hlkwd{ggplot}\hlstd{(crab.dat.zim,} \hlkwd{aes}\hlstd{(}\hlkwc{x}\hlstd{=x))}\hlopt{+}
          \hlkwd{geom_histogram}\hlstd{(}\hlkwd{aes}\hlstd{(}\hlkwc{y}\hlstd{=..density..),} \hlkwc{fill}\hlstd{=}\hlstr{"deepskyblue2"}\hlstd{,} \hlkwc{color}\hlstd{=}\hlstr{"black"}\hlstd{,}
                         \hlkwc{binwidth}\hlstd{=}\hlnum{1}\hlstd{)}\hlopt{+}
          \hlkwd{geom_linerange}\hlstd{(}\hlkwd{aes}\hlstd{(}\hlkwc{ymin}\hlstd{=}\hlnum{0}\hlstd{,} \hlkwc{ymax}\hlstd{=f),} \hlkwc{color}\hlstd{=}\hlstr{"dark red"}\hlstd{)}\hlopt{+}
          \hlkwd{scale_x_continuous}\hlstd{(}\hlstr{"Number of satellites"}\hlstd{,}
                             \hlkwc{labels} \hlstd{=} \hlkwd{as.character}\hlstd{(sats),} \hlkwc{breaks} \hlstd{= sats)}\hlopt{+}
          \hlkwd{labs}\hlstd{(}\hlkwc{y}\hlstd{=}\hlstr{"Density"}\hlstd{,}
               \hlkwc{title}\hlstd{=}\hlstr{"Number of observations per satellite"}\hlstd{,}
               \hlkwc{subtitle}\hlstd{=}\hlstr{"ZIP distribution. Lambda=5.4633205, Sigma=0.4550912"}\hlstd{)}\hlopt{+}
  \hlkwd{theme_bw}\hlstd{()}\hlopt{+}
  \hlkwd{geom_hline}\hlstd{(}\hlkwc{yintercept}\hlstd{=}\hlnum{0}\hlstd{)}
\end{alltt}
\end{kframe}
\includegraphics[width=\maxwidth]{figure/plot2-1} 
\end{knitrout}
\caption{ZIP with parameters that we calculated via MOM}
\label{plot2}
\end{center}
\end{figure}
It would seem MOM inflates zero values a bit too much and underestimates the other values. I would suggest that doing MLE would be a superior way of calculating parameters! Let's see if I am correct about it.

	 \item Find the maximum likelihood estimator for $\lambda$ and $\sigma$.\\
	 \textbf{Solution:}
\begin{knitrout}
\definecolor{shadecolor}{rgb}{0.969, 0.969, 0.969}\color{fgcolor}\begin{kframe}
\begin{alltt}
\hlcom{#MLE}
\hlstd{dpois.ll}\hlkwb{<-}\hlkwa{function}\hlstd{(}\hlkwc{par}\hlstd{,} \hlkwc{data}\hlstd{)\{}
  \hlstd{lambda} \hlkwb{<-} \hlstd{par[}\hlnum{1}\hlstd{]}
  \hlstd{pi} \hlkwb{<-} \hlstd{par[}\hlnum{2}\hlstd{]}
  \hlstd{ll} \hlkwb{<-} \hlkwd{sum}\hlstd{(}\hlkwd{dzipois}\hlstd{(}\hlkwc{x}\hlstd{=data,} \hlkwc{lambda}\hlstd{=lambda,} \hlkwc{pstr0}\hlstd{=pi,} \hlkwc{log}\hlstd{=T))}
  \hlopt{-}\hlstd{ll}
\hlstd{\}}
\hlstd{answerZIPll}\hlkwb{<-}\hlkwd{optim}\hlstd{(}\hlkwc{par} \hlstd{=} \hlkwd{c}\hlstd{(}\hlnum{1}\hlstd{,}\hlnum{0}\hlstd{),}
      \hlkwc{fn} \hlstd{= dpois.ll,}
      \hlkwc{data}\hlstd{=crab.dat}\hlopt{$}\hlstd{x)}
\hlkwd{paste}\hlstd{(}\hlstr{"Lambda:"}\hlstd{,answerZIPll}\hlopt{$}\hlstd{par[}\hlnum{1}\hlstd{],}\hlstr{"Sigma:"}\hlstd{,answerZIPll}\hlopt{$}\hlstd{par[}\hlnum{2}\hlstd{])}
\end{alltt}
\begin{verbatim}
## [1] "Lambda: 4.57709217492355 Sigma: 0.349638133663682"
\end{verbatim}
\end{kframe}
\end{knitrout}
	\item Plot a histogram of the data with the zero-inflated Poisson distribution
	fit with the MLE estimates. How well does the distribution fit the data?\\
	\textbf{Solution:} Now, let's assess if we were right about the ultimate superiority of MLE compared to MOM. Yet again, we're going to use VGAM package \cite{VGAM} to create a true density function.
\begin{knitrout}
\definecolor{shadecolor}{rgb}{0.969, 0.969, 0.969}\color{fgcolor}\begin{kframe}
\begin{alltt}
\hlstd{crab.dat.zim} \hlkwb{<-} \hlstd{crab.dat} \hlopt
  \hlkwd{mutate}\hlstd{(}\hlkwc{f} \hlstd{=} \hlkwd{dzipois}\hlstd{(}\hlkwc{x}\hlstd{=x,} \hlkwc{lambda}\hlstd{=}\hlnum{4.5770923}\hlstd{,} \hlkwc{pstr0}\hlstd{=}\hlnum{0.3496381}\hlstd{))}

\hlkwd{ggplot}\hlstd{(crab.dat.zim,} \hlkwd{aes}\hlstd{(}\hlkwc{x}\hlstd{=x))}\hlopt{+}
  \hlkwd{geom_histogram}\hlstd{(}\hlkwd{aes}\hlstd{(}\hlkwc{y}\hlstd{=..density..),} \hlkwc{fill}\hlstd{=}\hlstr{"deepskyblue2"}\hlstd{,} \hlkwc{color}\hlstd{=}\hlstr{"black"}\hlstd{,}
                 \hlkwc{binwidth}\hlstd{=}\hlnum{1}\hlstd{)}\hlopt{+}
  \hlkwd{geom_linerange}\hlstd{(}\hlkwd{aes}\hlstd{(}\hlkwc{ymin}\hlstd{=}\hlnum{0}\hlstd{,} \hlkwc{ymax}\hlstd{=f),} \hlkwc{color}\hlstd{=}\hlstr{"dark red"}\hlstd{)}\hlopt{+}
  \hlkwd{scale_x_continuous}\hlstd{(}\hlstr{"Number of satellites"}\hlstd{,}
                     \hlkwc{labels} \hlstd{=} \hlkwd{as.character}\hlstd{(sats),} \hlkwc{breaks} \hlstd{= sats)}\hlopt{+}
  \hlkwd{labs}\hlstd{(}\hlkwc{y}\hlstd{=}\hlstr{"Density"}\hlstd{,}
       \hlkwc{title}\hlstd{=}\hlstr{"Number of observations per satellite"}\hlstd{,}
       \hlkwc{subtitle}\hlstd{=}\hlstr{"ZIP distribution. Lambda=4.5770923, sigma=0.3496381"}\hlstd{)} \hlopt{+}
  \hlkwd{theme_bw}\hlstd{()}\hlopt{+}
  \hlkwd{geom_hline}\hlstd{(}\hlkwc{yintercept}\hlstd{=}\hlnum{0}\hlstd{)}
\end{alltt}
\end{kframe}
\includegraphics[width=\maxwidth]{figure/unnamed-chunk-6-1} 
\end{knitrout}
Now, that's definitely something to write home about! While I don't really like that this model undervalues some values (look at 1, for example), but it does really well when it comes to zero and most other values! \\
\textbf{Case closed:} MLE is superior! 
	\end{enumerate}
\newpage
%%%%%%%%%%%%%%%%%%%%%%%%%%%%%%%%%%%%%%%%%%%%%%%%%%%%%%%%%%%%%%%%%%%%%%%%%%%%%%%
%%%%%%%%%%%%%%%%%%%%%%%%%%%%%%%%%%%%%%%%%%%%%%%%%%%%%%%%%%%%%%%%%%%%%%%%%%%%%%%
%%%%%%%%%  Question 2
%%%%%%%%%%%%%%%%%%%%%%%%%%%%%%%%%%%%%%%%%%%%%%%%%%%%%%%%%%%%%%%%%%%%%%%%%%%%%%%
%%%%%%%%%%%%%%%%%%%%%%%%%%%%%%%%%%%%%%%%%%%%%%%%%%%%%%%%%%%%%%%%%%%%%%%%%%%%%%%
\item The time to death for rats injected with a toxic substance, denoted by $Y$
	(measured in days), follows an exponential distribution with $\lambda = 1/5$. That is,
	\[Y \sim \textrm{exponential}(\lambda = 1/5).\]
	This is the population distribution. It describes the time to death for all individual rats in the population.

  The \textbf{exponential distribution} serves as a very good model for measurements like waiting times or lifetimes.
  \begin{align*}
  \lambda &\in \mathbb{R^+}  \tag*{\textbf{[Parameters]}}\\
  \mathcal{X}&= \{\omega: \omega \in \mathbb{R^+}\} \tag*{\textbf{[Support]}}\\
  f_X(x|\mu,\sigma) &= \lambda e^{-\lambda x}~ I(x \in \mathbb{R^+}) \tag*{\textbf{[PDF]}}\\
  F_X(x|\mu,\sigma) &= \left(1- e^{-\lambda x}\right)I(x \in \mathbb{R^+}\tag*{\textbf{[CDF]}}\\
  F_X^{-1}(p|\mu,\sigma) &= \frac{-\ln(1-p)}{\lambda} \tag*{\textbf{[Inverse CDF]}}\\
  E(X) &= \frac{1}{\lambda} \tag*{\textbf{[Expected Value]}}\\
  var(X) &= \frac{1}{\lambda^2}\tag*{\textbf{[Population Variance]}}\\
  \end{align*}
	\begin{enumerate}
	  \item Plot the exponential$(\lambda = 1/5)$ population distribution.\\
	  \textbf{Solution}: I am going to use ggplot2 package \cite{ggplot} to plot this exponential distribution.
\begin{figure}[H]
\begin{center}
\begin{knitrout}
\definecolor{shadecolor}{rgb}{0.969, 0.969, 0.969}\color{fgcolor}\begin{kframe}
\begin{alltt}
\hlstd{x}\hlkwb{<-}\hlkwd{seq}\hlstd{(}\hlnum{0}\hlstd{,} \hlnum{100}\hlstd{,} \hlnum{0.1}\hlstd{)}
\hlstd{y}\hlkwb{<-}\hlkwd{dexp}\hlstd{(x,} \hlnum{1}\hlopt{/}\hlnum{5}\hlstd{)}
\hlstd{ggdat}\hlkwb{<-}\hlkwd{data.frame}\hlstd{(x,y)}

\hlkwd{ggplot}\hlstd{(ggdat,} \hlkwd{aes}\hlstd{(}\hlkwc{x}\hlstd{=x,} \hlkwc{y}\hlstd{=y))}\hlopt{+}
  \hlkwd{geom_line}\hlstd{(}\hlkwc{color}\hlstd{=}\hlstr{"dark red"}\hlstd{)}\hlopt{+}
  \hlkwd{labs}\hlstd{(}\hlkwc{y}\hlstd{=}\hlstr{"Density"}\hlstd{,}
  \hlkwc{x}\hlstd{=}\hlstr{"x"}\hlstd{,}
  \hlkwc{title}\hlstd{=}\hlstr{"Exponential distribution"}\hlstd{,}
  \hlkwc{subtitle}\hlstd{=}\hlstr{"Lambda=0.2"}\hlstd{)}\hlopt{+}
  \hlkwd{theme_bw}\hlstd{()}
\end{alltt}
\end{kframe}
\includegraphics[width=\maxwidth]{figure/unnamed-chunk-7-1} 
\end{knitrout}
\caption{Exponential distribution via \cite{ggplot}}
\label{p2plot1}
\end{center}
\end{figure}
	  \item Mathematical statisticians can show the exact sampling distributions 
	  of $\bar{Y}$ are gamma; i.e.,
	  \[\bar{Y} \sim \textrm{gamma}(\alpha=n,\beta=\frac{1}{n\lambda}).\]
	  The gamma distribution is described below.
\begin{align*}
\alpha &> 0, \beta>0 \tag*{\textbf{[Parameters]}}\\
\mathcal{X} = (0, \infty)\tag*{\textbf{[Support]}}\\
f_X(x) &= \frac{1}{\beta^\alpha \Gamma(\alpha)} x^{\alpha-1} e^{-x/\beta} I(x>0)
\tag*{\textbf{[PDF]}}\\
E(X) = \alpha \beta \tag*{\textbf{[Population Mean]}}\\
var(X) = \alpha \beta^2 \tag*{\textbf{[Population Variance]}}
\end{align*}
You can ask \texttt{R} for the gamma distribution PDF using 
\texttt{dgamma(x=x,shape=alpha,scale=beta)}.
Plot the exact sampling distribution for $n=2$, $n=10$, $n=35$, and
$n=50$.\\
\textbf{Solution:} Now, in order to plot our gamme distribution, we need to use \texttt{dgamma()} function within the base R package! Let's see how the graphs are going to vary based on the changes to $n$!
\begin{knitrout}
\definecolor{shadecolor}{rgb}{0.969, 0.969, 0.969}\color{fgcolor}\begin{kframe}
\begin{alltt}
\hlkwd{library}\hlstd{(patchwork)}
\hlcom{#plotting distributions based on the info you provided!}
\hlstd{ggdat} \hlkwb{<-} \hlkwd{data.frame}\hlstd{(}\hlkwc{x1}\hlstd{=}\hlkwd{seq}\hlstd{(}\hlnum{0}\hlstd{,} \hlnum{100}\hlstd{,} \hlnum{0.1}\hlstd{))}\hlopt
  \hlkwd{mutate}\hlstd{(}\hlkwc{y1} \hlstd{=} \hlkwd{dgamma}\hlstd{(}\hlkwc{x}\hlstd{=x1,} \hlkwc{shape}\hlstd{=}\hlnum{2}\hlstd{,} \hlkwc{scale}\hlstd{=}\hlnum{1}\hlopt{/}\hlstd{(}\hlnum{2}\hlopt{*}\hlnum{0.2}\hlstd{)),}
         \hlkwc{y2} \hlstd{=} \hlkwd{dgamma}\hlstd{(}\hlkwc{x}\hlstd{=x1,} \hlkwc{shape}\hlstd{=}\hlnum{10}\hlstd{,} \hlkwc{scale}\hlstd{=}\hlnum{1}\hlopt{/}\hlstd{(}\hlnum{10}\hlopt{*}\hlnum{0.2}\hlstd{)),}
         \hlkwc{y3} \hlstd{=} \hlkwd{dgamma}\hlstd{(}\hlkwc{x}\hlstd{=x1,} \hlkwc{shape}\hlstd{=}\hlnum{35}\hlstd{,} \hlkwc{scale}\hlstd{=}\hlnum{1}\hlopt{/}\hlstd{(}\hlnum{35}\hlopt{*}\hlnum{0.2}\hlstd{)),}
         \hlkwc{y4} \hlstd{=} \hlkwd{dgamma}\hlstd{(}\hlkwc{x}\hlstd{=x1,} \hlkwc{shape}\hlstd{=}\hlnum{50}\hlstd{,} \hlkwc{scale}\hlstd{=}\hlnum{1}\hlopt{/}\hlstd{(}\hlnum{50}\hlopt{*}\hlnum{0.2}\hlstd{)))}

\hlcom{#You would ask me:}
\hlcom{#Chris, but why do you limit your x from 0 to 30.}
\hlcom{#And I would tell you:}
\hlcom{#Because it's cool! }
\hlstd{plot1}\hlkwb{<-}\hlkwd{ggplot}\hlstd{(ggdat,} \hlkwd{aes}\hlstd{(}\hlkwc{x}\hlstd{=x,} \hlkwc{y}\hlstd{=y1))}\hlopt{+}
  \hlkwd{geom_line}\hlstd{(}\hlkwc{color}\hlstd{=}\hlstr{"green"}\hlstd{)}\hlopt{+}
  \hlkwd{xlim}\hlstd{(}\hlnum{0}\hlstd{,}\hlnum{30}\hlstd{)}\hlopt{+}
  \hlkwd{theme_bw}\hlstd{()}\hlopt{+}
  \hlkwd{labs}\hlstd{(}\hlkwc{y}\hlstd{=}\hlstr{"Density"}\hlstd{,}
       \hlkwc{title}\hlstd{=}\hlstr{"Gamma distribution for n = 2"}\hlstd{,}
       \hlkwc{subtitle}\hlstd{=}\hlstr{"Alpha = 2, Beta = 2.5"}\hlstd{)}

\hlstd{plot2}\hlkwb{<-}\hlkwd{ggplot}\hlstd{(ggdat,} \hlkwd{aes}\hlstd{(}\hlkwc{x}\hlstd{=x,} \hlkwc{y}\hlstd{=y2))}\hlopt{+}
  \hlkwd{geom_line}\hlstd{(}\hlkwc{color}\hlstd{=}\hlstr{"dark red"}\hlstd{)}\hlopt{+}
  \hlkwd{xlim}\hlstd{(}\hlnum{0}\hlstd{,}\hlnum{30}\hlstd{)}\hlopt{+}
  \hlkwd{theme_bw}\hlstd{()}\hlopt{+}
  \hlkwd{labs}\hlstd{(}\hlkwc{y}\hlstd{=}\hlstr{"Density"}\hlstd{,}
       \hlkwc{title}\hlstd{=}\hlstr{"Gamma distribution for n = 10"}\hlstd{,}
       \hlkwc{subtitle}\hlstd{=}\hlstr{"Alpha = 10, Beta = 0.5"}\hlstd{)}

\hlstd{plot3}\hlkwb{<-}\hlkwd{ggplot}\hlstd{(ggdat,} \hlkwd{aes}\hlstd{(}\hlkwc{x}\hlstd{=x,} \hlkwc{y}\hlstd{=y3))}\hlopt{+}
  \hlkwd{geom_line}\hlstd{(}\hlkwc{color}\hlstd{=}\hlstr{"blue"}\hlstd{)}\hlopt{+}
  \hlkwd{xlim}\hlstd{(}\hlnum{0}\hlstd{,}\hlnum{30}\hlstd{)}\hlopt{+}
  \hlkwd{theme_bw}\hlstd{()}\hlopt{+}
  \hlkwd{labs}\hlstd{(}\hlkwc{y}\hlstd{=}\hlstr{"Density"}\hlstd{,}
       \hlkwc{title}\hlstd{=}\hlstr{"Gamma distribution for n = 35"}\hlstd{,}
       \hlkwc{subtitle}\hlstd{=}\hlstr{"Alpha = 35, Beta = 0.14"}\hlstd{)}

\hlstd{plot4}\hlkwb{<-}\hlkwd{ggplot}\hlstd{(ggdat,} \hlkwd{aes}\hlstd{(}\hlkwc{x}\hlstd{=x,} \hlkwc{y}\hlstd{=y4))}\hlopt{+}
  \hlkwd{geom_line}\hlstd{(}\hlkwc{color}\hlstd{=}\hlstr{"black"}\hlstd{)}\hlopt{+}
  \hlkwd{xlim}\hlstd{(}\hlnum{0}\hlstd{,}\hlnum{30}\hlstd{)}\hlopt{+}
  \hlkwd{theme_bw}\hlstd{()}\hlopt{+}
  \hlkwd{labs}\hlstd{(}\hlkwc{y}\hlstd{=}\hlstr{"Density"}\hlstd{,}
       \hlkwc{title}\hlstd{=}\hlstr{"Gamma distribution for n = 50"}\hlstd{,}
       \hlkwc{subtitle}\hlstd{=}\hlstr{"Alpha = 50, Beta = 0.1"}\hlstd{)}

\hlstd{(plot1}\hlopt{|}\hlstd{plot2)}\hlopt{/}\hlstd{(plot3}\hlopt{|}\hlstd{plot4)}
\end{alltt}
\end{kframe}
\end{knitrout}

\begin{figure}[H]
\begin{center}
\begin{knitrout}
\definecolor{shadecolor}{rgb}{0.969, 0.969, 0.969}\color{fgcolor}
\includegraphics[width=\maxwidth]{figure/unnamed-chunk-8-1} 
\end{knitrout}
\caption{Gamma distributions for $n=2$, $n=10$, $n=35$, and $n=50$}
\label{p2plot3}
\end{center}
\end{figure}

It seems to me that the bigger $n$ gets, the more normal gamma distribution becomes (yet another proof that CLT is correct!). To prove that I am correct, I am going to superimpose a normal distribution function.
	  \item The Central Limit Theorem says that as $n$ increases, the sampling 
	  distribution of $\bar{Y}$ can be well approximated with a Gaussian 
	  distribution. Superimpose the approximate sampling distribution of $\bar{Y}$
	  for $n=2$, $n=10$, $n=35$, and $n=50$.\\
\textbf{Solution:} So in order to apply CLT here, we need to keep in mind that $\mu_{\bar{x}}=\mu_{x}$ and $\sigma_{\bar{x}}=sqrt(var(\bar{X}))$. Let's apply it when plotting!
\begin{knitrout}
\definecolor{shadecolor}{rgb}{0.969, 0.969, 0.969}\color{fgcolor}\begin{kframe}
\begin{alltt}
\hlstd{plot1} \hlkwb{<-} \hlstd{plot1}\hlopt{+}
  \hlkwd{geom_function}\hlstd{(}\hlkwc{fun}\hlstd{=dnorm,} \hlkwc{args}\hlstd{=}\hlkwd{list}\hlstd{(}\hlkwc{mean}\hlstd{=}\hlnum{2}\hlopt{*}\hlstd{(}\hlnum{1}\hlopt{/}\hlstd{(}\hlnum{2}\hlopt{*}\hlnum{0.2}\hlstd{)),} \hlkwc{sd}\hlstd{=}\hlkwd{sqrt}\hlstd{(}\hlnum{2}\hlopt{*}\hlstd{(}\hlnum{1}\hlopt{/}\hlstd{(}\hlnum{2}\hlopt{*}\hlnum{0.2}\hlstd{)}\hlopt{^}\hlnum{2}\hlstd{))),}
                \hlkwc{color}\hlstd{=}\hlstr{"red"}\hlstd{)}

\hlstd{plot2} \hlkwb{<-} \hlstd{plot2}\hlopt{+}
  \hlkwd{geom_function}\hlstd{(}\hlkwc{fun}\hlstd{=dnorm,} \hlkwc{args}\hlstd{=}\hlkwd{list}\hlstd{(}\hlkwc{mean}\hlstd{=}\hlnum{10}\hlopt{*}\hlstd{(}\hlnum{1}\hlopt{/}\hlstd{(}\hlnum{10}\hlopt{*}\hlnum{0.2}\hlstd{)),} \hlkwc{sd}\hlstd{=}\hlkwd{sqrt}\hlstd{(}\hlnum{10}\hlopt{*}\hlstd{(}\hlnum{1}\hlopt{/}\hlstd{(}\hlnum{10}\hlopt{*}\hlnum{0.2}\hlstd{)}\hlopt{^}\hlnum{2}\hlstd{))),}
                \hlkwc{color}\hlstd{=}\hlstr{"red"}\hlstd{)}

\hlstd{plot3} \hlkwb{<-} \hlstd{plot3}\hlopt{+}
  \hlkwd{geom_function}\hlstd{(}\hlkwc{fun}\hlstd{=dnorm,} \hlkwc{args}\hlstd{=}\hlkwd{list}\hlstd{(}\hlkwc{mean}\hlstd{=}\hlnum{35}\hlopt{*}\hlstd{(}\hlnum{1}\hlopt{/}\hlstd{(}\hlnum{35}\hlopt{*}\hlnum{0.2}\hlstd{)),} \hlkwc{sd}\hlstd{=}\hlkwd{sqrt}\hlstd{(}\hlnum{35}\hlopt{*}\hlstd{(}\hlnum{1}\hlopt{/}\hlstd{(}\hlnum{35}\hlopt{*}\hlnum{0.2}\hlstd{)}\hlopt{^}\hlnum{2}\hlstd{))),}
                \hlkwc{color}\hlstd{=}\hlstr{"red"}\hlstd{)}

\hlstd{plot4} \hlkwb{<-} \hlstd{plot4}\hlopt{+}
  \hlkwd{geom_function}\hlstd{(}\hlkwc{fun}\hlstd{=dnorm,} \hlkwc{args}\hlstd{=}\hlkwd{list}\hlstd{(}\hlkwc{mean}\hlstd{=}\hlnum{50}\hlopt{*}\hlstd{(}\hlnum{1}\hlopt{/}\hlstd{(}\hlnum{50}\hlopt{*}\hlnum{0.2}\hlstd{)),} \hlkwc{sd}\hlstd{=}\hlkwd{sqrt}\hlstd{(}\hlnum{50}\hlopt{*}\hlstd{(}\hlnum{1}\hlopt{/}\hlstd{(}\hlnum{50}\hlopt{*}\hlnum{0.2}\hlstd{))}\hlopt{^}\hlnum{2}\hlstd{)),}
                \hlkwc{color}\hlstd{=}\hlstr{"red"}\hlstd{)}

\hlstd{(plot1}\hlopt{|}\hlstd{plot2)}\hlopt{/}\hlstd{(plot3}\hlopt{|}\hlstd{plot4)}
\end{alltt}
\end{kframe}
\end{knitrout}
\begin{figure}[H]
\begin{center}
\begin{knitrout}
\definecolor{shadecolor}{rgb}{0.969, 0.969, 0.969}\color{fgcolor}
\includegraphics[width=\maxwidth]{figure/unnamed-chunk-9-1} 
\end{knitrout}
\caption{Gamma distributions for $n=2$, $n=10$, $n=35$, and $n=50$ with superimposed Gaussian distribution}
\label{p2plot4}
\end{center}
\end{figure}
	  \item Find the probability that a randomly selected rat injected with the toxic substance
	  lives between 1 and 3 days.\\
	  \textbf{Solution:}
\begin{knitrout}
\definecolor{shadecolor}{rgb}{0.969, 0.969, 0.969}\color{fgcolor}\begin{kframe}
\begin{alltt}
\hlcom{# 1-P(x<=1)-(1-P(x>3))}
\hlnum{1}\hlopt{-}\hlkwd{pgamma}\hlstd{(}\hlkwc{q}\hlstd{=}\hlnum{1}\hlstd{,} \hlkwc{shape}\hlstd{=}\hlnum{1}\hlstd{,} \hlkwc{scale}\hlstd{=}\hlnum{1}\hlopt{/}\hlnum{0.2}\hlstd{)}\hlopt{-}\hlstd{(}\hlnum{1}\hlopt{-}\hlkwd{pgamma}\hlstd{(}\hlkwc{q}\hlstd{=}\hlnum{3}\hlstd{,} \hlkwc{shape}\hlstd{=}\hlnum{1}\hlstd{,} \hlkwc{scale}\hlstd{=}\hlnum{1}\hlopt{/}\hlnum{0.2}\hlstd{))}
\end{alltt}
\begin{verbatim}
## [1] 0.2699191
\end{verbatim}
\end{kframe}
\end{knitrout}
	  \item Find the exact probability, using the exact sampling distribution, that two randomly selected rats injected with the toxic substance live  between 1 and 3 days on average.\\
	  \textbf{Solution:}
\begin{knitrout}
\definecolor{shadecolor}{rgb}{0.969, 0.969, 0.969}\color{fgcolor}\begin{kframe}
\begin{alltt}
\hlcom{# 1-P(x<=1)-(1-P(x>3))}
\hlnum{1}\hlopt{-}\hlkwd{pgamma}\hlstd{(}\hlkwc{q}\hlstd{=}\hlnum{1}\hlstd{,} \hlkwc{shape}\hlstd{=}\hlnum{2}\hlstd{,} \hlkwc{scale}\hlstd{=}\hlnum{1}\hlopt{/}\hlstd{(}\hlnum{2}\hlopt{*}\hlnum{0.2}\hlstd{))}\hlopt{-}
  \hlstd{(}\hlnum{1}\hlopt{-}\hlkwd{pgamma}\hlstd{(}\hlkwc{q}\hlstd{=}\hlnum{3}\hlstd{,} \hlkwc{shape}\hlstd{=}\hlnum{2}\hlstd{,} \hlkwc{scale}\hlstd{=}\hlnum{1}\hlopt{/}\hlstd{(}\hlnum{2}\hlopt{*}\hlnum{0.2}\hlstd{)))}
\end{alltt}
\begin{verbatim}
## [1] 0.2758208
\end{verbatim}
\end{kframe}
\end{knitrout}
  \item Find the approximate probability, using the Central Limit Theorem, that two randomly 
	  selected rats injected with the toxic substance live between 1 and 3 days on average. Comment
	  on connection between the results and the assumptions of Central Limit Theorem.\\
	  \textbf{Solution:}
\begin{knitrout}
\definecolor{shadecolor}{rgb}{0.969, 0.969, 0.969}\color{fgcolor}\begin{kframe}
\begin{alltt}
\hlnum{1}\hlopt{-}\hlkwd{pnorm}\hlstd{(}\hlnum{1}\hlstd{,} \hlkwc{mean}\hlstd{=}\hlnum{2}\hlopt{*}\hlstd{(}\hlnum{1}\hlopt{/}\hlstd{(}\hlnum{2}\hlopt{*}\hlnum{0.2}\hlstd{)),} \hlkwc{sd}\hlstd{=}\hlkwd{sqrt}\hlstd{(}\hlnum{2}\hlopt{*}\hlstd{(}\hlnum{1}\hlopt{/}\hlstd{(}\hlnum{2}\hlopt{*}\hlnum{0.2}\hlstd{)}\hlopt{^}\hlnum{2}\hlstd{)))}\hlopt{-}
  \hlstd{(}\hlnum{1}\hlopt{-}\hlkwd{pnorm}\hlstd{(}\hlnum{3}\hlstd{,} \hlkwc{mean}\hlstd{=}\hlnum{2}\hlopt{*}\hlstd{(}\hlnum{1}\hlopt{/}\hlstd{(}\hlnum{2}\hlopt{*}\hlnum{0.2}\hlstd{)),} \hlkwc{sd}\hlstd{=}\hlkwd{sqrt}\hlstd{(}\hlnum{2}\hlopt{*}\hlstd{(}\hlnum{1}\hlopt{/}\hlstd{(}\hlnum{2}\hlopt{*}\hlnum{0.2}\hlstd{)}\hlopt{^}\hlnum{2}\hlstd{))))}
\end{alltt}
\begin{verbatim}
## [1] 0.1568543
\end{verbatim}
\end{kframe}
\end{knitrout}
We can see that the probability is far off. The explanation is simple: here, we are using only two rats (so $n=2$), and one of CLT assumptions is that $n>30$. So no wonder that the result is off! 
\item Under what conditions would the approximate probability calculated in 
part (f) better match the exact probability in part (e)?\\
\textbf{Solution:} According to the CLT, the bigger the $n$, the closer our AG distribution will get to the actual distribution.
  \end{enumerate}
\newpage
%%%%%%%%%%%%%%%%%%%%%%%%%%%%%%%%%%%%%%%%%%%%%%%%%%%%%%%%%%%%%%%%%%%%%%%%%%%%%%%
%%%%%%%%%%%%%%%%%%%%%%%%%%%%%%%%%%%%%%%%%%%%%%%%%%%%%%%%%%%%%%%%%%%%%%%%%%%%%%%
%%%%%%%%%  Question 3
%%%%%%%%%%%%%%%%%%%%%%%%%%%%%%%%%%%%%%%%%%%%%%%%%%%%%%%%%%%%%%%%%%%%%%%%%%%%%%%
%%%%%%%%%%%%%%%%%%%%%%%%%%%%%%%%%%%%%%%%%%%%%%%%%%%%%%%%%%%%%%%%%%%%%%%%%%%%%%%
\item Below you will load and summarize a dataset 
  containing 575 observations of drug treatments. The data includes the following
  \begin{itemize}
    \item ID --	Identification Code	(1 - 575)
    \item AGE	-- Age at Enrollment	(Years)
    \item BECK -- Beck Depression Score	(0.000 - 54.000)
    \item HC --	Heroin/Cocaine Use During	3 Months Prior to Admission (1 = Heroin
    \& Cocaine; 2 = Heroin Only, 3 = Cocaine Only; 4 = Neither Heroin nor Cocaine)
    \item IV -- History of IV Drug Use	(1 = Never; 2 = Previous; 3 = Recent)
    \item IV3	-- Recent IV use	(1 = Yes; 0 = No)
    \item NDT -- Number of Prior Drug Treatments (0 - 40)
    \item RACE -- Subject's Race	(0 = White; 1 = Non-White)
\item TREAT -- Treatment Randomization (0 = Short Assignment;	1 = Long Assignment)
\item SITE -- Treatment Site (0 = A; 1 = B)
\item LEN.T	-- Length of Stay in Treatment (Days Admission Date to Exit Date)	
\item TIME -- Time to Drug Relapse (Days Measured from Admission Date)
\item CENSOR -- Event for Treating Lost to Follow-Up as Returned to Drugs 
(1 = Returned to Drugs or Lost to Follow-Up; 0 = Otherwise)
\item etc.
\end{itemize}
\begin{enumerate} %this begins a lettered enumerate so I can ask more 
%than one question in a question.
\item Load the data provided in the ``quantreg" package for \texttt{R} \citep{quantreg}.
\begin{knitrout}
\definecolor{shadecolor}{rgb}{0.969, 0.969, 0.969}\color{fgcolor}\begin{kframe}
\begin{alltt}
\hlkwd{library}\hlstd{(}\hlstr{"quantreg"}\hlstd{)}
\hlkwd{data}\hlstd{(}\hlstr{"uis"}\hlstd{)}
\end{alltt}
\end{kframe}
\end{knitrout}
\item Is there evidence that patients receiving drug treatments are at least mildly depressed
on average? That is, is there evidence that the average BECK depression score is greater
than 13, $\mu>13$?
    \begin{enumerate}
      \item What is the null hypothesis for this test?\\
      \textbf{Solution:} $H_{0}: \mu=13$
      \item What is the alternative hypothesis for this test?\\
      \textbf{Solution:} $H_{a}: \mu>13$
      \item What is the sample mean BECK score for these data?\\
      \textbf{Solution:}
\begin{knitrout}
\definecolor{shadecolor}{rgb}{0.969, 0.969, 0.969}\color{fgcolor}\begin{kframe}
\begin{alltt}
\hlstd{mu} \hlkwb{<-} \hlkwd{mean}\hlstd{(uis}\hlopt{$}\hlstd{BECK)}
\hlstd{mu0} \hlkwb{<-} \hlnum{13}
\hlkwd{paste}\hlstd{(}\hlstr{"The mean of BECK is"}\hlstd{,} \hlkwd{mean}\hlstd{(mu))}
\end{alltt}
\begin{verbatim}
## [1] "The mean of BECK is 17.367427826087"
\end{verbatim}
\end{kframe}
\end{knitrout}
      \item What is the test statistics for this test?\\
      \textbf{Solution:} Since we are working with the means here, going for the T-score would be the most reasonable solution, so we are going to use T-test. 
      \item At what value of $\bar{X}$ does the rejection region start for $\alpha=0.05$?\\

      \begin{align*}
	se &= \frac{s}{\sqrt{n}} \tag{Formula for finding t-value}\\
	t &= \frac{\bar{X}-\mu}{se} \tag{Formula for finding t-value}\\
	\bar{X} &= t*se + \mu \tag{Solving for $\bar{X}$}
	    \end{align*}
\begin{knitrout}
\definecolor{shadecolor}{rgb}{0.969, 0.969, 0.969}\color{fgcolor}\begin{kframe}
\begin{alltt}
\hlstd{n}\hlkwb{<-}\hlkwd{nrow}\hlstd{(uis)}
\hlstd{se} \hlkwb{=} \hlkwd{sd}\hlstd{(uis}\hlopt{$}\hlstd{BECK)}\hlopt{/}\hlkwd{sqrt}\hlstd{(n)}
\hlstd{(value}\hlkwb{<-}\hlstd{(}\hlkwd{qt}\hlstd{(}\hlnum{.95}\hlstd{, n}\hlopt{-}\hlnum{1}\hlstd{)}\hlopt{*}\hlstd{se)}\hlopt{+}\hlstd{mu0)}
\end{alltt}
\begin{verbatim}
## [1] 13.64123
\end{verbatim}
\end{kframe}
\end{knitrout}
      \item What is the p value for this test?
\begin{knitrout}
\definecolor{shadecolor}{rgb}{0.969, 0.969, 0.969}\color{fgcolor}\begin{kframe}
\begin{alltt}
\hlstd{n}\hlkwb{=}\hlkwd{nrow}\hlstd{(uis)}
\hlstd{tstat} \hlkwb{<-} \hlkwd{t.test}\hlstd{(}\hlkwc{x}\hlstd{=uis}\hlopt{$}\hlstd{BECK,}
       \hlkwc{mu} \hlstd{=} \hlnum{13}\hlstd{,}
       \hlkwc{alternative} \hlstd{=} \hlstr{"greater"}\hlstd{)}
\hlkwd{paste}\hlstd{(}\hlstr{"P-value:"}\hlstd{, tstat}\hlopt{$}\hlstd{p.value)}
\end{alltt}
\begin{verbatim}
## [1] "P-value: 7.41212268149766e-27"
\end{verbatim}
\end{kframe}
\end{knitrout}
      \item Graph the results of this test.\\
      \textbf{Solution:} Since we are using T-test for this, let's create a T-distribution with 574 degrees of freedom. Then, we are going to map our rejection region onto the graph (it's going to be at $t=1.647513$). Finally, I am going to put our value on X-axis and assess how far away it is from the rejection region (keep in mind the p-value — we already have in mind enough evidence to reject null hypothesis).
\begin{figure}[H]
\begin{center}
\begin{knitrout}
\definecolor{shadecolor}{rgb}{0.969, 0.969, 0.969}\color{fgcolor}\begin{kframe}
\begin{alltt}
\hlstd{ggdat} \hlkwb{<-} \hlkwd{data.frame}\hlstd{(}\hlkwc{t}\hlstd{=}\hlkwd{seq}\hlstd{(}\hlopt{-}\hlnum{5}\hlstd{,}\hlnum{5}\hlstd{,}\hlkwc{length}\hlstd{=}\hlnum{500}\hlstd{))}\hlopt
  \hlkwd{mutate}\hlstd{(}\hlkwc{f}\hlstd{=}\hlkwd{dt}\hlstd{(}\hlkwc{x}\hlstd{=t,} \hlkwc{df}\hlstd{=n}\hlopt{-}\hlnum{1}\hlstd{))}

\hlkwd{ggplot}\hlstd{(}\hlkwc{data}\hlstd{=ggdat,} \hlkwd{aes}\hlstd{(}\hlkwc{x}\hlstd{=t,} \hlkwc{y}\hlstd{=f))}\hlopt{+}
  \hlkwd{geom_line}\hlstd{()} \hlopt{+}
  \hlkwd{geom_hline}\hlstd{(}\hlkwc{yintercept}\hlstd{=}\hlnum{0}\hlstd{)}\hlopt{+}
  \hlkwd{geom_point}\hlstd{(}\hlkwd{aes}\hlstd{(}\hlkwc{x}\hlstd{=tstat}\hlopt{$}\hlstd{statistic,} \hlkwc{y}\hlstd{=}\hlnum{0}\hlstd{),} \hlkwc{color}\hlstd{=}\hlstr{"red"}\hlstd{)}\hlopt{+}
  \hlkwd{geom_vline}\hlstd{(}\hlkwc{xintercept}\hlstd{=}\hlkwd{qt}\hlstd{(}\hlkwc{p}\hlstd{=}\hlnum{0.95}\hlstd{,} \hlkwc{df}\hlstd{=n}\hlopt{-}\hlnum{1}\hlstd{),}
             \hlkwc{linetype}\hlstd{=}\hlstr{"dashed"}\hlstd{,} \hlkwc{color}\hlstd{=}\hlstr{"red"}\hlstd{)}\hlopt{+}
  \hlkwd{annotate}\hlstd{(}\hlstr{"text"}\hlstd{,} \hlkwc{x}\hlstd{=}\hlnum{1.2}\hlopt{*}\hlkwd{qt}\hlstd{(}\hlkwc{p}\hlstd{=}\hlnum{0.95}\hlstd{,} \hlkwc{df}\hlstd{=n}\hlopt{-}\hlnum{1}\hlstd{),} \hlkwc{y}\hlstd{=}\hlnum{0.30}\hlstd{,}
           \hlkwc{label}\hlstd{=}\hlstr{"Rejection Region"}\hlstd{,} \hlkwc{angle}\hlstd{=}\hlstr{"270"}\hlstd{,}
           \hlkwc{color}\hlstd{=}\hlstr{"red"}\hlstd{)}\hlopt{+}
  \hlkwd{annotate}\hlstd{(}\hlstr{"text"}\hlstd{,} \hlkwc{x}\hlstd{=tstat}\hlopt{$}\hlstd{statistic}\hlopt{-}\hlnum{1.8}\hlstd{,} \hlkwc{y}\hlstd{=}\hlnum{0.02}\hlstd{,}
           \hlkwc{label}\hlstd{=}\hlstr{"P-value = 7.41 * 10^-27"}\hlstd{,}
           \hlkwc{size}\hlstd{=}\hlnum{3}\hlstd{)}\hlopt{+}
  \hlkwd{theme_bw}\hlstd{()}\hlopt{+}
  \hlkwd{labs}\hlstd{(}\hlkwc{x}\hlstd{=}\hlstr{"T"}\hlstd{,}
       \hlkwc{y}\hlstd{=}\hlstr{"Density"}\hlstd{,}
       \hlkwc{title}\hlstd{=}\hlstr{"T-distribution to assess a null hypothesis"}\hlstd{)}
\end{alltt}
\end{kframe}
\includegraphics[width=\maxwidth]{figure/p3plot1-1} 
\end{knitrout}
\caption{T-distribution with 574 degrees of freedom}
\label{p3plot1}
\end{center}
\end{figure}
      \item Report a 95\% confidence interval for the average BECK depression score
and interpret it in the context of this question.\\
\textbf{Solution:}
\begin{knitrout}
\definecolor{shadecolor}{rgb}{0.969, 0.969, 0.969}\color{fgcolor}\begin{kframe}
\begin{alltt}
\hlstd{ci}\hlkwb{<-}\hlkwd{t.test}\hlstd{(}\hlkwc{x}\hlstd{=uis}\hlopt{$}\hlstd{BECK,}
       \hlkwc{alternative} \hlstd{=} \hlstr{"two.sided"}\hlstd{)}

\hlkwd{paste}\hlstd{(}\hlstr{"The confidence interval is ["}\hlstd{, ci}\hlopt{$}\hlstd{conf.int[}\hlnum{1}\hlstd{],} \hlstr{", "}\hlstd{,}
      \hlstd{ci}\hlopt{$}\hlstd{conf.int[}\hlnum{2}\hlstd{],}\hlstr{"]"}\hlstd{,} \hlkwc{sep}\hlstd{=}\hlstr{""}\hlstd{)}
\end{alltt}
\begin{verbatim}
## [1] "The confidence interval is [16.6029755170859, 18.131880135088]"
\end{verbatim}
\end{kframe}
\end{knitrout}
    \end{enumerate}
\item Is there a significant difference in the length of stay in treatment by 
    treatment site?\\
\textbf{Solution:} First, let's establish our hypothesis.
\begin{align*}
H_{0}&:\mu_{0}-\mu{1}=0\\
H_{a}&:\mu_{0}-\mu{1}\neq0
\end{align*}
\begin{knitrout}
\definecolor{shadecolor}{rgb}{0.969, 0.969, 0.969}\color{fgcolor}\begin{kframe}
\begin{alltt}
\hlstd{res} \hlkwb{<-} \hlkwd{t.test}\hlstd{(uis}\hlopt{$}\hlstd{LEN.T} \hlopt{~} \hlstd{uis}\hlopt{$}\hlstd{SITE,}
              \hlkwc{data} \hlstd{= uis,} \hlkwc{var.equal} \hlstd{= F)}
\hlkwd{paste}\hlstd{(}\hlstr{"P-value:"}\hlstd{,res}\hlopt{$}\hlstd{p.value)}
\end{alltt}
\begin{verbatim}
## [1] "P-value: 4.07192587806883e-09"
\end{verbatim}
\end{kframe}
\end{knitrout}
Based on this two-sample t-test, we have enough evidence to suggest that (with 95\% confidence) our true mean is not equal to 0! Therefore, we have statistically significant results showing that there's a difference in the length of stay between two sites!
  \end{enumerate}
  \newpage
%%%%%%%%%%%%%%%%%%%%%%%%%%%%%%%%%%%%%%%%%%%%%%%%%%%%%%%%%%%%%%%%%%%%%%%%%%%%%%%
%%%%%%%%%%%%%%%%%%%%%%%%%%%%%%%%%%%%%%%%%%%%%%%%%%%%%%%%%%%%%%%%%%%%%%%%%%%%%%%
%%%%%%%%%  Question 4
%%%%%%%%%%%%%%%%%%%%%%%%%%%%%%%%%%%%%%%%%%%%%%%%%%%%%%%%%%%%%%%%%%%%%%%%%%%%%%%
%%%%%%%%%%%%%%%%%%%%%%%%%%%%%%%%%%%%%%%%%%%%%%%%%%%%%%%%%%%%%%%%%%%%%%%%%%%%%%%
\item Below you will load and summarize a dataset containing 53 observations of prostate cance patients.
In this research, a number of possible predictor variables were measured before surgery. The patients then 
had surgery to determine nodal involvement. 
  \begin{itemize}
    \item r -- Nodal Involvement (0=No, 1=Yes)
    \item aged	-- Age Group (0=Less than 60, 1=At least 60)
    \item stage -- Palpitation Result Severity (0=Less severe, 1=More severe)
    \item grade -- Biopsy Result Severity (0=Less severe, 1=More severe)
    \item xray -- X-ray Result Severity (0=Less severe, 1=More severe)
    \item acid -- the level of acid phosphatase in the blood serum
\end{itemize}
The treatment strategy for a patient diagnosed with cancer of the prostate depend highly on whether the cancer has spread to the surrounding lymph nodes (nodal involvement). It is common to operate on the patient to get samples from the nodes which can then be analysed under a microscope.

\begin{enumerate} %this begins a lettered enumerate so I can ask more 
%than one question in a question.
\item Load the data provided in the ``boot" package for \texttt{R} \citep{boot}.
\begin{knitrout}
\definecolor{shadecolor}{rgb}{0.969, 0.969, 0.969}\color{fgcolor}\begin{kframe}
\begin{alltt}
\hlkwd{library}\hlstd{(}\hlstr{"boot"}\hlstd{)}
\hlkwd{data}\hlstd{(}\hlstr{"nodal"}\hlstd{)}
\end{alltt}
\end{kframe}
\end{knitrout}
\item Is there evidence that less than half of prostate cancer patients have nodal involvment?
    \begin{enumerate}
      \item What is the null hypothesis for this test?\\
      \textbf{Solution:}  $H_{0}: \hat{P}=0.5$
      \item What is the alternative hypothesis for this test?\\
      \textbf{Solution:}  $H_{0}: \hat{P}<0.5$
      \item What is the sample proportion of patients with nodal involvement for these data?\\
      \textbf{Solution:} It's 37\%!
\begin{knitrout}
\definecolor{shadecolor}{rgb}{0.969, 0.969, 0.969}\color{fgcolor}\begin{kframe}
\begin{alltt}
\hlkwd{table}\hlstd{(nodal}\hlopt{$}\hlstd{r)[}\hlnum{2}\hlstd{]}\hlopt{/}\hlkwd{nrow}\hlstd{(nodal)} \hlcom{#37%}
\end{alltt}
\begin{verbatim}
##         1 
## 0.3773585
\end{verbatim}
\end{kframe}
\end{knitrout}
      \item What is the test statistics for this test?\\
      \textbf{Solution:} We are working with proportions, so we are going to use a Z-value.
      \item At what value of $\hat{P}$ does the rejection region start for $\alpha=0.05$?\\
      \textbf{Solution:}
      \begin{align*}
      z &= \frac{\hat{p}-p_{0}}{\sqrt{\frac{p_{0}(1-p_{0})}{n}}}\\
      \hat{p} &= \frac{\sqrt{n}z\sqrt{p_{0}(1-p_{0})}}{n} + p_{0}
      \end{align*}
      
\begin{knitrout}
\definecolor{shadecolor}{rgb}{0.969, 0.969, 0.969}\color{fgcolor}\begin{kframe}
\begin{alltt}
\hlstd{p0} \hlkwb{<-} \hlnum{0.5}
\hlstd{phat} \hlkwb{<-} \hlkwd{table}\hlstd{(nodal}\hlopt{$}\hlstd{r)[}\hlnum{2}\hlstd{]}\hlopt{/}\hlkwd{nrow}\hlstd{(nodal)}
\hlstd{n} \hlkwb{<-} \hlkwd{nrow}\hlstd{(nodal)}
\hlstd{z1} \hlkwb{<-} \hlkwd{qnorm}\hlstd{(}\hlnum{.05}\hlstd{,} \hlnum{0}\hlstd{,} \hlnum{1}\hlstd{)}

\hlstd{answerProp}\hlkwb{<-}\hlstd{(}\hlkwd{sqrt}\hlstd{(n)}\hlopt{*}\hlstd{z1}\hlopt{*}\hlkwd{sqrt}\hlstd{(p0}\hlopt{*}\hlstd{(}\hlnum{1}\hlopt{-}\hlstd{p0)))}\hlopt{/}\hlstd{(n)} \hlopt{+} \hlstd{p0}

\hlkwd{paste}\hlstd{(}\hlstr{"Rejection starts with"}\hlstd{, answerProp,} \hlstr{"but we already have a lower number:"}\hlstd{,}
      \hlkwd{table}\hlstd{(nodal}\hlopt{$}\hlstd{r)[}\hlnum{2}\hlstd{]}\hlopt{/}\hlkwd{nrow}\hlstd{(nodal))}
\end{alltt}
\begin{verbatim}
## [1] "Rejection starts with 0.38703098909445 but we already have a lower number: 0.377358490566038"
\end{verbatim}
\end{kframe}
\end{knitrout}
Having a number lower than the rejection region is the first sign that true $\hat{P}$ is not 0.5!
      \item What is the p value for this test?
\begin{knitrout}
\definecolor{shadecolor}{rgb}{0.969, 0.969, 0.969}\color{fgcolor}\begin{kframe}
\begin{alltt}
\hlstd{answerPropTest} \hlkwb{<-} \hlkwd{prop.test}\hlstd{(}\hlnum{20}\hlstd{,} \hlnum{53}\hlstd{,} \hlkwc{p}\hlstd{=}\hlnum{0.5}\hlstd{,} \hlkwc{alternative}\hlstd{=}\hlstr{"less"}\hlstd{)}
\hlkwd{paste}\hlstd{(}\hlstr{"P-value:"}\hlstd{,answerPropTest}\hlopt{$}\hlstd{p.value)}
\end{alltt}
\begin{verbatim}
## [1] "P-value: 0.0496428173856788"
\end{verbatim}
\end{kframe}
\end{knitrout}
We have enough evidence to reject null hypothesis in favor of the alternative!
      \item Graph the results of this test.
\begin{figure}[H]
\begin{center}
\begin{knitrout}
\definecolor{shadecolor}{rgb}{0.969, 0.969, 0.969}\color{fgcolor}\begin{kframe}
\begin{alltt}
\hlstd{ggdat} \hlkwb{<-} \hlkwd{data.frame}\hlstd{(}\hlkwc{t}\hlstd{=}\hlkwd{seq}\hlstd{(}\hlopt{-}\hlnum{5}\hlstd{,}\hlnum{5}\hlstd{,}\hlkwc{length}\hlstd{=}\hlnum{500}\hlstd{))}\hlopt
  \hlkwd{mutate}\hlstd{(}\hlkwc{f}\hlstd{=}\hlkwd{dnorm}\hlstd{(}\hlkwc{x}\hlstd{=t,} \hlkwc{mean}\hlstd{=}\hlnum{0}\hlstd{,} \hlkwc{sd}\hlstd{=}\hlnum{1}\hlstd{))}

\hlkwd{ggplot}\hlstd{(}\hlkwc{data}\hlstd{=ggdat,} \hlkwd{aes}\hlstd{(}\hlkwc{x}\hlstd{=t,} \hlkwc{y}\hlstd{=f))}\hlopt{+}
  \hlkwd{geom_line}\hlstd{()} \hlopt{+}
  \hlkwd{geom_hline}\hlstd{(}\hlkwc{yintercept}\hlstd{=}\hlnum{0}\hlstd{)}\hlopt{+}
   \hlkwd{geom_point}\hlstd{(}\hlkwd{aes}\hlstd{(}\hlkwc{x}\hlstd{=}\hlopt{-}\hlkwd{sqrt}\hlstd{(answerPropTest}\hlopt{$}\hlstd{statistic),} \hlkwc{y}\hlstd{=}\hlnum{0}\hlstd{),} \hlkwc{color}\hlstd{=}\hlstr{"red"}\hlstd{)}\hlopt{+}
  \hlkwd{geom_vline}\hlstd{(}\hlkwc{xintercept}\hlstd{=}\hlkwd{qnorm}\hlstd{(}\hlkwc{p}\hlstd{=}\hlnum{0.05}\hlstd{,} \hlkwc{mean}\hlstd{=}\hlnum{0}\hlstd{,} \hlkwc{sd}\hlstd{=}\hlnum{1}\hlstd{),}
              \hlkwc{linetype}\hlstd{=}\hlstr{"dashed"}\hlstd{,} \hlkwc{color}\hlstd{=}\hlstr{"red"}\hlstd{)}\hlopt{+}
  \hlkwd{annotate}\hlstd{(}\hlstr{"text"}\hlstd{,} \hlkwc{x}\hlstd{=}\hlnum{1.2}\hlopt{*}\hlkwd{qnorm}\hlstd{(}\hlkwc{p}\hlstd{=}\hlnum{0.05}\hlstd{,} \hlkwc{mean}\hlstd{=}\hlnum{0}\hlstd{,} \hlkwc{sd}\hlstd{=}\hlnum{1}\hlstd{),} \hlkwc{y}\hlstd{=}\hlnum{0.30}\hlstd{,}
           \hlkwc{label}\hlstd{=}\hlstr{"Rejection Region"}\hlstd{,} \hlkwc{angle}\hlstd{=}\hlstr{"90"}\hlstd{,}
           \hlkwc{color}\hlstd{=}\hlstr{"red"}\hlstd{)}\hlopt{+}
  \hlkwd{annotate}\hlstd{(}\hlstr{"text"}\hlstd{,} \hlopt{-}\hlkwd{sqrt}\hlstd{(answerPropTest}\hlopt{$}\hlstd{statistic),} \hlkwc{y}\hlstd{=}\hlnum{0.01}\hlstd{,}
           \hlkwc{label}\hlstd{=}\hlstr{"p-value = 0.04964"}\hlstd{,}
           \hlkwc{size}\hlstd{=}\hlnum{3}\hlstd{)}\hlopt{+}
  \hlkwd{theme_bw}\hlstd{()}\hlopt{+}
  \hlkwd{labs}\hlstd{(}\hlkwc{x}\hlstd{=}\hlstr{"X"}\hlstd{,}
       \hlkwc{y}\hlstd{=}\hlstr{"Density"}\hlstd{,}
       \hlkwc{title}\hlstd{=}\hlstr{"Z-test for proportion"}\hlstd{)}
\end{alltt}
\end{kframe}
\includegraphics[width=\maxwidth]{figure/unnamed-chunk-23-1} 
\end{knitrout}
\caption{Results of Z-test}
\label{p4plot1}
\end{center}
\end{figure}
It's one of those times when it's really hard to tell by the graph alone if the dot within the rejection region or outside of it. So, as I've done before, we can use \texttt{prop.test()} in order to check the p-value!
      \item Report a 95\% confidence interval for the proportion of prostate cancer patients with nodal involvement and interpret it in the context of this question.
\begin{knitrout}
\definecolor{shadecolor}{rgb}{0.969, 0.969, 0.969}\color{fgcolor}\begin{kframe}
\begin{alltt}
\hlstd{propCI}\hlkwb{<-}\hlkwd{prop.test}\hlstd{(}\hlnum{20}\hlstd{,} \hlnum{53}\hlstd{,} \hlkwc{alternative}\hlstd{=}\hlstr{"two.sided"}\hlstd{)}
\hlkwd{paste}\hlstd{(}\hlstr{"Confidence interval is"}\hlstd{, propCI}\hlopt{$}\hlstd{conf.int)}
\end{alltt}
\begin{verbatim}
## [1] "Confidence interval is 0.251167223511435"
## [2] "Confidence interval is 0.521281347288513"
\end{verbatim}
\end{kframe}
\end{knitrout}
    \end{enumerate}
    \item Clearly, it would be preferable if an accurate assessment of nodal involvement could be made without surgery. Is there a significant difference in the nodal involvement of patients with any of the severity indicators?\\
    \textbf{Solution:}\\
    $H_{0}:\hat{p}_{1}-\hat{p}_{2}=0$\\
    $H_{0}:\hat{p}_{1}-\hat{p}_{2}\neq0$
\begin{knitrout}
\definecolor{shadecolor}{rgb}{0.969, 0.969, 0.969}\color{fgcolor}\begin{kframe}
\begin{alltt}
\hlcom{#H0: p1-p2 = 0}
\hlcom{#Ha: p1-p2 != 0}
\hlstd{nodal.c} \hlkwb{<-} \hlstd{nodal} \hlopt
  \hlkwd{mutate}\hlstd{(}\hlkwc{sev} \hlstd{=} \hlkwd{case_when}\hlstd{(stage}\hlopt{==}\hlnum{1} \hlopt{~} \hlnum{1}\hlstd{,}
                         \hlstd{grade}\hlopt{==}\hlnum{1} \hlopt{~} \hlnum{1}\hlstd{,}
                         \hlstd{xray}\hlopt{==}\hlnum{1} \hlopt{~} \hlnum{1}\hlstd{,}
                         \hlnum{TRUE} \hlopt{~} \hlnum{0}\hlstd{))} \hlcom{#put all severity indicators into one}
\hlcom{#non severe}
\hlstd{x1}\hlkwb{<-}\hlnum{1}
\hlstd{n1}\hlkwb{<-}\hlnum{16}
\hlcom{#Non severe: 1 involvement out of 16 cases}

\hlcom{#severe}
\hlstd{x2}\hlkwb{<-}\hlnum{19}
\hlstd{n2}\hlkwb{<-}\hlnum{37}
\hlcom{#Severe: 19 involvements out of 37 cases}

\hlstd{finalProp} \hlkwb{<-} \hlkwd{prop.test}\hlstd{(}\hlkwc{x}\hlstd{=}\hlkwd{c}\hlstd{(x1,x2),}
          \hlkwc{n}\hlstd{=}\hlkwd{c}\hlstd{(n1,n2))}
\hlkwd{paste}\hlstd{(}\hlstr{"P-value of two-sample test is"}\hlstd{, finalProp}\hlopt{$}\hlstd{p.value)}
\end{alltt}
\begin{verbatim}
## [1] "P-value of two-sample test is 0.00509373051672405"
\end{verbatim}
\end{kframe}
\end{knitrout}
Based on the p-value, we have enough evidence to reject the null hypothesis in favor of alternative. Therefore, there's actually a significant in the nodal involvement in patients with severity indicators. 
  \end{enumerate}
\end{enumerate}
%%%%%%%%%%%%%%%%%%%%%%%%%%%%%%%%%%%%%%%%%%%%%%%%%%%%%%%%%%%%%%%%%%%%%%%%%%%%%%%
%%%%%%%%%%%%%%%%%%%%%%%%%%%%%%%%%%%%%%%%%%%%%%%%%%%%%%%%%%%%%%%%%%%%%%%%%%%%%%%
%%%%%%%%%  Bibliography
%%%%%%%%%%%%%%%%%%%%%%%%%%%%%%%%%%%%%%%%%%%%%%%%%%%%%%%%%%%%%%%%%%%%%%%%%%%%%%%
%%%%%%%%%%%%%%%%%%%%%%%%%%%%%%%%%%%%%%%%%%%%%%%%%%%%%%%%%%%%%%%%%%%%%%%%%%%%%%%
\newpage
\noindent \textbf{Bonus 1:} Use the gganimate package \citep{gganimate} for \texttt{R} to create a plot that demonstrates the Central Limit Theorem for the Poisson, Binomial, Exponential, and Gaussian distributions in a $2\times 2$ grid as \href{https://moodle.colgate.edu/mod/forum/discuss.php?d=127682}{here}.\\

Note that we can't add GIFs to the .pdf document, so you'll have to email me your code for this part. You'll find \texttt{  transition\_time()} helpful for creating your animation and \texttt{gganimate\_save()} helpful for saving your animation. \vspace{2em}\\
\textbf{Solution:}
\begin{knitrout}
\definecolor{shadecolor}{rgb}{0.969, 0.969, 0.969}\color{fgcolor}\begin{kframe}
\begin{alltt}
\hlkwd{library}\hlstd{(gganimate)}
\hlkwd{library}\hlstd{(gifski)}
\hlkwd{library}\hlstd{(transformr)}

\hlcom{#Binom}
\hlcom{#E(X)=np}
\hlcom{#Var(X)=np*(1-p)}
\hlcom{#100*0.5*(1-0.5)}

\hlstd{final1.df} \hlkwb{<-} \hlkwd{c}\hlstd{()}
\hlkwa{for}\hlstd{(num} \hlkwa{in} \hlnum{1}\hlopt{:}\hlnum{150}\hlstd{)\{}
  \hlstd{generate} \hlkwb{<-} \hlkwd{rnorm}\hlstd{(num,} \hlkwc{mean}\hlstd{=num}\hlopt{*}\hlnum{0.3}\hlstd{,} \hlkwc{sd}\hlstd{=}\hlkwd{sqrt}\hlstd{(num}\hlopt{*}\hlnum{0.3}\hlopt{*}\hlnum{0.7}\hlstd{))}
  \hlstd{num.df} \hlkwb{<-} \hlkwd{data.frame}\hlstd{(}\hlkwc{values}\hlstd{=generate,} \hlkwc{n}\hlstd{=num)}
  \hlstd{final1.df}\hlkwb{<-}\hlkwd{bind_rows}\hlstd{(final1.df, num.df)}
\hlstd{\}}
\hlstd{p1}\hlkwb{<-}\hlkwd{ggplot}\hlstd{(final1.df,} \hlkwd{aes}\hlstd{(}\hlkwc{x}\hlstd{=values))}\hlopt{+}
  \hlkwd{geom_histogram}\hlstd{(}\hlkwd{aes}\hlstd{(}\hlkwc{y}\hlstd{=..density..),}
                 \hlkwc{color}\hlstd{=}\hlstr{"white"}\hlstd{,}
                 \hlkwc{fill}\hlstd{=}\hlstr{"dark red"}\hlstd{)}\hlopt{+}
  \hlkwd{geom_function}\hlstd{(}\hlkwc{fun}\hlstd{=dnorm,} \hlkwc{args}\hlstd{=}\hlkwd{list}\hlstd{(}\hlkwc{mean}\hlstd{=num}\hlopt{*}\hlnum{0.3}\hlstd{,} \hlkwc{sd}\hlstd{=}\hlkwd{sqrt}\hlstd{(num}\hlopt{*}\hlnum{0.3}\hlopt{*}\hlnum{0.7}\hlstd{)),}
                \hlkwc{color}\hlstd{=}\hlstr{"red"}\hlstd{)}\hlopt{+}
  \hlkwd{labs}\hlstd{(}\hlkwc{x}\hlstd{=}\hlstr{"X"}\hlstd{,}
       \hlkwc{y}\hlstd{=}\hlstr{"Density"}\hlstd{,}
       \hlkwc{title}\hlstd{=}\hlstr{"Poisson distribution"}\hlstd{,}
       \hlkwc{subtitle}\hlstd{=}\hlstr{"Lambda: 5"}\hlstd{)}\hlopt{+}
  \hlkwd{transition_states}\hlstd{(n,}
                    \hlkwc{transition_length} \hlstd{=} \hlnum{1}\hlstd{,}
                    \hlkwc{state_length} \hlstd{=} \hlnum{1}\hlstd{)}\hlopt{+}
  \hlkwd{ease_aes}\hlstd{(}\hlstr{'sine-in-out'}\hlstd{)}

\hlstd{anim1} \hlkwb{<-} \hlkwd{animate}\hlstd{(p1,} \hlkwc{renderer} \hlstd{=} \hlkwd{gifski_renderer}\hlstd{())}
\hlkwd{anim_save}\hlstd{(}\hlstr{"poisson.gif"}\hlstd{, anim1)}


\hlcom{#Poisson}
\hlcom{#E(X)=lambda}
\hlcom{#Var(X)=lambda}
\hlstd{final2.df} \hlkwb{<-} \hlkwd{c}\hlstd{()}
\hlkwa{for}\hlstd{(num} \hlkwa{in} \hlnum{1}\hlopt{:}\hlnum{150}\hlstd{)\{}
  \hlstd{generate} \hlkwb{<-} \hlkwd{rnorm}\hlstd{(num,} \hlkwc{mean}\hlstd{=}\hlnum{5}\hlstd{,} \hlkwc{sd}\hlstd{=}\hlkwd{sqrt}\hlstd{(}\hlnum{5}\hlstd{))}
  \hlstd{num.df} \hlkwb{<-} \hlkwd{data.frame}\hlstd{(}\hlkwc{values}\hlstd{=generate,} \hlkwc{n}\hlstd{=num)}
  \hlstd{final2.df}\hlkwb{<-}\hlkwd{bind_rows}\hlstd{(final2.df, num.df)}
\hlstd{\}}
\hlstd{p2}\hlkwb{<-}\hlkwd{ggplot}\hlstd{(final2.df,} \hlkwd{aes}\hlstd{(}\hlkwc{x}\hlstd{=values))}\hlopt{+}
  \hlkwd{geom_histogram}\hlstd{(}\hlkwd{aes}\hlstd{(}\hlkwc{y}\hlstd{=..density..),}
                 \hlkwc{color}\hlstd{=}\hlstr{"white"}\hlstd{,}
                 \hlkwc{fill}\hlstd{=}\hlstr{"dark red"}\hlstd{)}\hlopt{+}
  \hlkwd{geom_function}\hlstd{(}\hlkwc{fun}\hlstd{=dnorm,} \hlkwc{args}\hlstd{=}\hlkwd{list}\hlstd{(}\hlkwc{mean}\hlstd{=}\hlnum{5}\hlstd{,} \hlkwc{sd}\hlstd{=}\hlkwd{sqrt}\hlstd{(}\hlnum{5}\hlstd{)),}
                \hlkwc{color}\hlstd{=}\hlstr{"red"}\hlstd{)}\hlopt{+}
  \hlkwd{labs}\hlstd{(}\hlkwc{x}\hlstd{=}\hlstr{"X"}\hlstd{,}
       \hlkwc{y}\hlstd{=}\hlstr{"Density"}\hlstd{,}
       \hlkwc{title}\hlstd{=}\hlstr{"Poisson distribution"}\hlstd{,}
       \hlkwc{subtitle}\hlstd{=}\hlstr{"Lambda: 5"}\hlstd{)}\hlopt{+}
  \hlkwd{transition_states}\hlstd{(n,}
                    \hlkwc{transition_length} \hlstd{=} \hlnum{1}\hlstd{,}
                    \hlkwc{state_length} \hlstd{=} \hlnum{1}\hlstd{)}\hlopt{+}
  \hlkwd{ease_aes}\hlstd{(}\hlstr{'sine-in-out'}\hlstd{)}

\hlstd{anim2} \hlkwb{<-} \hlkwd{animate}\hlstd{(p2,} \hlkwc{renderer} \hlstd{=} \hlkwd{gifski_renderer}\hlstd{())}
\hlkwd{anim_save}\hlstd{(}\hlstr{"poisson.gif"}\hlstd{, anim2)}


\hlcom{#Exponential}
\hlcom{#E(X)=1/lambda}
\hlcom{#Var(X)=1/lambda^2}

\hlstd{final3.df} \hlkwb{<-} \hlkwd{c}\hlstd{()}
\hlkwa{for}\hlstd{(num} \hlkwa{in} \hlnum{1}\hlopt{:}\hlnum{150}\hlstd{)\{}
  \hlstd{generate} \hlkwb{<-} \hlkwd{rnorm}\hlstd{(num,} \hlkwc{mean}\hlstd{=}\hlnum{1}\hlopt{/}\hlnum{5}\hlstd{,} \hlkwc{sd}\hlstd{=}\hlkwd{sqrt}\hlstd{(}\hlnum{1}\hlopt{/}\hlnum{25}\hlstd{))}
  \hlstd{num.df} \hlkwb{<-} \hlkwd{data.frame}\hlstd{(}\hlkwc{values}\hlstd{=generate,} \hlkwc{n}\hlstd{=num)}
  \hlstd{final3.df}\hlkwb{<-}\hlkwd{bind_rows}\hlstd{(final3.df, num.df)}
\hlstd{\}}

\hlstd{p3}\hlkwb{<-}\hlkwd{ggplot}\hlstd{(final3.df,} \hlkwd{aes}\hlstd{(}\hlkwc{x}\hlstd{=values))}\hlopt{+}
  \hlkwd{geom_histogram}\hlstd{(}\hlkwd{aes}\hlstd{(}\hlkwc{y}\hlstd{=..density..),}
                 \hlkwc{color}\hlstd{=}\hlstr{"white"}\hlstd{,}
                 \hlkwc{fill}\hlstd{=}\hlstr{"dark red"}\hlstd{)}\hlopt{+}
  \hlkwd{geom_function}\hlstd{(}\hlkwc{fun}\hlstd{=dnorm,} \hlkwc{args}\hlstd{=}\hlkwd{list}\hlstd{(}\hlkwc{mean}\hlstd{=}\hlnum{1}\hlopt{/}\hlnum{5}\hlstd{,} \hlkwc{sd}\hlstd{=}\hlkwd{sqrt}\hlstd{(}\hlnum{1}\hlopt{/}\hlnum{25}\hlstd{)),}
                \hlkwc{color}\hlstd{=}\hlstr{"red"}\hlstd{)}\hlopt{+}
  \hlkwd{labs}\hlstd{(}\hlkwc{x}\hlstd{=}\hlstr{"X"}\hlstd{,}
       \hlkwc{y}\hlstd{=}\hlstr{"Density"}\hlstd{,}
       \hlkwc{title}\hlstd{=}\hlstr{"Exponential distribution"}\hlstd{,}
       \hlkwc{subtitle}\hlstd{=}\hlstr{"Lambda: 5"}\hlstd{)}\hlopt{+}
\hlkwd{transition_states}\hlstd{(n,}
                    \hlkwc{transition_length} \hlstd{=} \hlnum{1}\hlstd{,}
                    \hlkwc{state_length} \hlstd{=} \hlnum{1}\hlstd{)}\hlopt{+}
  \hlkwd{ease_aes}\hlstd{(}\hlstr{'sine-in-out'}\hlstd{)}

\hlstd{anim3} \hlkwb{<-} \hlkwd{animate}\hlstd{(p3,} \hlkwc{renderer} \hlstd{=} \hlkwd{gifski_renderer}\hlstd{())}
\hlkwd{anim_save}\hlstd{(}\hlstr{"exponential.gif"}\hlstd{, anim3)}

\hlcom{#Normal}
\hlcom{#E(X)=mu}
\hlcom{#Var(X)=sigma^2}
\hlstd{final4.df} \hlkwb{<-} \hlkwd{c}\hlstd{()}
\hlkwa{for}\hlstd{(num} \hlkwa{in} \hlnum{1}\hlopt{:}\hlnum{150}\hlstd{)\{}
  \hlstd{generate} \hlkwb{<-} \hlkwd{rnorm}\hlstd{(num,} \hlkwc{mean}\hlstd{=}\hlnum{0}\hlstd{,} \hlkwc{sd}\hlstd{=}\hlnum{1}\hlstd{)}
  \hlstd{num.df} \hlkwb{<-} \hlkwd{data.frame}\hlstd{(}\hlkwc{values}\hlstd{=generate,} \hlkwc{n}\hlstd{=num)}
  \hlstd{final4.df}\hlkwb{<-}\hlkwd{bind_rows}\hlstd{(final4.df, num.df)}
\hlstd{\}}
\hlstd{p4}\hlkwb{<-}\hlkwd{ggplot}\hlstd{(final4.df,} \hlkwd{aes}\hlstd{(}\hlkwc{x}\hlstd{=values))}\hlopt{+}
  \hlkwd{geom_histogram}\hlstd{(}\hlkwd{aes}\hlstd{(}\hlkwc{y}\hlstd{=..density..),}
                 \hlkwc{color}\hlstd{=}\hlstr{"white"}\hlstd{,}
                 \hlkwc{fill}\hlstd{=}\hlstr{"dark red"}\hlstd{)}\hlopt{+}
  \hlkwd{geom_function}\hlstd{(}\hlkwc{fun}\hlstd{=dnorm,} \hlkwc{args}\hlstd{=}\hlkwd{list}\hlstd{(}\hlkwc{mean}\hlstd{=}\hlnum{1}\hlopt{/}\hlnum{5}\hlstd{,} \hlkwc{sd}\hlstd{=}\hlkwd{sqrt}\hlstd{(}\hlnum{1}\hlopt{/}\hlnum{25}\hlstd{)),}
                \hlkwc{color}\hlstd{=}\hlstr{"red"}\hlstd{)}\hlopt{+}
  \hlkwd{labs}\hlstd{(}\hlkwc{x}\hlstd{=}\hlstr{"X"}\hlstd{,}
       \hlkwc{y}\hlstd{=}\hlstr{"Density"}\hlstd{,}
       \hlkwc{title}\hlstd{=}\hlstr{"Normal distribution"}\hlstd{,}
       \hlkwc{subtitle}\hlstd{=}\hlstr{"Mean: 0, SD: 1"}\hlstd{)}\hlopt{+}
  \hlkwd{transition_states}\hlstd{(n,}
                    \hlkwc{transition_length} \hlstd{=} \hlnum{1}\hlstd{,}
                    \hlkwc{state_length} \hlstd{=} \hlnum{1}\hlstd{)}\hlopt{+}
  \hlkwd{ease_aes}\hlstd{(}\hlstr{'sine-in-out'}\hlstd{)}

\hlstd{anim4} \hlkwb{<-} \hlkwd{animate}\hlstd{(p4,} \hlkwc{renderer} \hlstd{=} \hlkwd{gifski_renderer}\hlstd{())}
\hlkwd{anim_save}\hlstd{(}\hlstr{"normal.gif"}\hlstd{, anim4)}
\end{alltt}
\end{kframe}
\end{knitrout}

\noindent \textbf{Bonus 2:} Compare the effectiveness of the $t$-interval with the bootstrap interval. In a loop, generate 1000 datasets, evaluate a $t$ and bootstrapping confidence interval for each set of data, and track whether you've captured the true population mean. An effective answer here would evaluate this several times varying sample size and the data generating distribution.

\newpage
\bibliography{bib}
\end{document}
